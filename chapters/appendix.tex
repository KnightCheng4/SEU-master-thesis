\appendix
\newtheorem{theorem}{定理}

\chapter{欧几里得第二定理的证明}
	\begin{theorem}
		欧几里得第二定理(素数有无穷多个)\\
		证明:用反证法。假设素数有有限个($N$个),记为$p_1,p_2,\dots,p_N$。则我们构造一个新的数,
		
		\[n=p_1p_2\dots p_N+1.\] 
		
		由于$p_i,i=1,2,\dots,N$为素数,则一定不为$1$。于是对于任意的$p_i,i=1,2,\dots, N$,有 
		
		\[p_i\not|n\] 
		
		这表明,要么$n$本身为素数,要么$n$为合数,但是存在$p_1,p_2,\dots,p_N$之外的其他素数能够将$n$进行素因子分解。不管哪种情况,都表明存在更多的素数。定理得证。\qed
	\end{theorem}

\chapter{$\sqrt{2}$是无理数的证明}
	\begin{theorem}
		$\sqrt{2}$是无理数。\\
		证明:用反证法。假设$\sqrt{2}$是有理数,则可表示为两个整数的商,即$\exists p,q, q\ne0$ 
		
		\[\sqrt{2}=\frac{p}{q}\] 
		
		不失一般性,我们假设$p,q$是既约的,即$\gcd(p,q)=1$。对上式两边平方可得\\
		
		\begin{align*}
			2& =\frac{p^2}{q^2}\\
			p^2&=2q^2.
		\end{align*}
		
		表明$p^2$为偶数,因此$p$为偶数,记$p=2m$。则
		
		\begin{align*}
			p^2&=4m^2=2q^2\\
			q^2&=2m^2.
		\end{align*}
		
		表明$q$也为偶数,因此它们有公共因子$2$。这与它们既约的假设矛盾。定理得证。\qed
	\end{theorem}