\chapter{绪论}
	
	\section{研究背景}
		Android提供了具有权限控制的安全系统。但是,应用程序开发人员可以在应用程序(a.k.a应用程序)中创建漏洞,因为除了应用程序所需的权限信息之外,它还设置了太多权限。为了解决这个问题,已经对字节码级别的权限控制和方法级安全功能进行了研究\cite{Backes_appguard_2013}\cite{zhang_efficient_2014}\cite{Neisse_privacy_2016}\cite{Lee_appwrapping_2017}。由于权限控制方法控制与权限相关的所有API,因此即使在不需要控制的部分中也生成API控制,从而导致开销。此外,由于它控制与危险API相关的权限,因此存在执行权限控制的缺点,而不管应用程序的操作流程如何。此外,由于基于API的控制,无法添加和控制应用程序的方法级安全功能。此外,当应用程序正在使用时,会显示权限控制窗口,这会降低用户的便利性。
		
		在字节码级别添加方法级安全性的先前方法\cite{Lee_appwrapping_2017}是使用appwrapping在字节码级别添加适当的安全功能,其中不安全应用程序需要安全功能。此方法在通过反编译缺少安全功能的应用程序(不安全应用程序)获得的字节码(smali文件)中需要安全性的地方重新打包(重新编译和签名)适当的安全功能。但是,根据静态策略,只在必要时添加单个安全功能。此外,每次更改安全功能时,都应添加并应用新的安全功能。因此,每当添加了安全功能的应用程序的安全策略发生更改时,必须添加安全功能并重新打包,从而使策略管理变得困难且用户友好性降低。
		
	\section{国内外研究现状}
		Android提供依赖权限控制的安全系统;但是,它需要的权限多于应用程序所需的最低权限,并且很难将每个权限的API数量微调到平均七个API\cite{Felt_android_2011}。为了解决这些漏洞,权限控制研究为用户提供了一种控制与可利用或具有隐私问题的API相关的权限的方法。
		
		Zhang等提出了一种静态分析安全漏洞应用程序的方法\cite{zhang_efficient_2014}。该方法简要描述如下:提前检查与权限相关的API可能泄漏敏感信息的位置,并将权限控制代码添加到这些位置。权限控制允许用户通过显示警报窗口来选择是允许还是拒绝权限。该方法需要静态分析要执行的app的初步任务,并且存在仅可以控制与许可相关的API的限制。此外,通过强制最终用户确定是否允许该许可来丢失UI的便利性。

		Backes等为应用程序添加了监控代码以进行权限控制\cite{Backes_appguard_2013}。该代码监视整个应用程序中发生的与权限相关的API调用,并将其与预配置的策略文件进行比较。当调用与策略文件中声明的权限相关的API时,它会引发警报窗口。警报窗口使用户可以选择显示多少信息。例如,在位置信息的情况下,可以选择是否提供近似位置信息或准确位置信息。然而,这种方法的缺点是必须实时操作监视服务以进行许可控制。此外,如在先前的研究中,可以仅控制与许可相关的API,并且要求用户决定是否允许减少用户便利性的许可。

		Neisse等人提出了一种基于预先配置的权限控制策略来控制权限的方法\cite{Neisse_privacy_2016}。当API调用具有潜在的隐私威胁时,该策略将设置要提供的信息级别。设置策略是权限控制库始终处于活动状态,始终监视API调用,并在存在相应调用时调整根据策略提供的信息级别。它为九个项目提供用户界面,包括结构,操作,威胁规则和策略设置角色。但是,无论应用程序的流程如何,都可以为隐私威胁API设置策略。根据应用程序的进度,可能无法将安全功能添加到需要安全性的位置。

		还有一些研究在字节码级别的方法单元上需要安全性的位置添加安全性功能,而不是权限控制。Lee等人提出了一种解决方案,使用appwrapping技术在方法上添加必要的安全功能,该技术在没有应用程序的Android源代码的情况下在字节码级别添加安全功能\cite{Lee_appwrapping_2017}。这包括将所需的安全功能提取到小字代码中,这些代码是字节码级别,并添加安全功能以解决所需方法位置缺乏安全性的问题。但是,它基于静态策略运行,每次更改策略时,都会重复安全功能提取,添加和应用程序;此外,在更改策略时需要重新打包。另外,缺点在于必须事先知道app的进度以便适当地应用策略。
	
	\section{论文研究内容}
	
		在本文中,我们提出了一个动态的基于策略的自动AppWrapper工具包,它可以在需要字节码级别安全性的位置为不安全的应用程序添加安全功能,并管理安全功能,而无需通过动态策略管理进行重新打包。建议的AppWrapper工具包的目的如下:

		\begin{enumerate}
			\item 在方法级别添加安全功能:在不安全应用程序的所有活动类的方法单元中添加安全功能执行代码。可执行代码使用Java反射技术调用安全性函数。添加的安全功能可执行代码根据动态策略执行。
			\item 动态策略管理:在方法单元中添加安全功能执行代码后,通过检查设置策略的策略文件和执行安全功能执行代码的位置来确定安全功能的执行(方法单元)活动类)。该策略具有安全功能API以及将在其中执行安全功能的应用程序的位置(类和方法名称)以及要执行的安全功能。只需更改策略,即可在不重新打包应用程序的情况下应用更改的策略。
			\item 实时应用程序操作流程检查和策略设置:通过在方法级别添加实时应用程序日志功能,用户/管理员可以在运行应用程序时实时检查日志并了解应用程序的进度流程(类和方法名称)。在运行应用程序时,用户必须同时检查应用程序的当前类和方法名称,并将安全功能添加到需要安全性的方法的位置。
		\end{enumerate}

	\section{论文组织结构}
				
		本文的其余部分如下。
		
		在第2节中,总结了相关的研究和研究。
		
		在第3节中,描述了提议的框架。
		
		在第4节中,给出了提出的框架的性能分析的实验结果。
		
		第5节讨论了几个问题。
		
		最后,我们在第6节中提出我们的结论。