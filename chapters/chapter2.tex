\chapter{相关技术研究}
	
	本章介绍现代手机操作系统的权限控制,具体包括基本的安全机制、Android系统和iOS系统的权限管理以及它们对内置传感器的权限控制策略。

	\section{智能终端权限控制}
		
		Android和iOS分别是基于Linux内核和Unix内核进行设计的现代操作系统,它们继承了类Unix操作系统的基础安全框架,又在此基础上各自发展出了一些新的安全措施和隐私保护机制。现代手机操作系统安全机制的一个重要功能就是权限管理。通过设定访问设备中各种软硬件资源的权限需求、权限组和权限等级,操作系统就能够将软硬件资源的访问牢牢地控制住。在多年的攻防实践中,系统安全工程师们不断地修补漏洞,构筑更完善的安全体系。在经过长足的发展后,现在的手机操作系统安全机制已经基本成熟,能够预防一定强度的网络攻击,保护用户的隐私信息。
		
		因此,想要实现对用户输入内容的推测或窃取,就必须对手机操作系统的安全机制和防护体系有具体的认识。只有通过分析和理解手机操作系统的安全机制和权限控制原理,才能找到合适的信息源辅助数据窃取和分析工作。本节首先简要介绍手机操作系统的安全机制,然后分别对Android和iOS两个系统的权限控制机制进行详细介绍,最后说明这两个操作系统对内置动作传感器的权限控制。
		
		\subsection{操作系统安全机制}
		
			Android操作系统采用一种名为SandBox(沙盒)的机制来进行进程隔离和应用隔离。一般情况下,各个运行中的应用程序都以独立进程运行在各自的虚拟机中。每个应用程序在首次安装时都会被分配一个UserID,这个ID在全局中是唯一固定且保持不变的。并且,该UserID与Linux内核中的进程UID用户名一一对应。而手机中运行的不同进程无法访问其他进程的软硬件资源,也无法和其他程序共享数据。因此,除了相关进程的程序外,其他应用程序是不能直接获得用户在屏幕软键盘上的输入的,这就有效地防止了恶意程序对用户隐私的窥探。
			
			iOS系统的隔离机制则更为激进,它采用了被称为“岛式存储”的存储结构。Android系统虽然运行中的程序无法共享资源,但还存在统一的资源管理器,由系统维护着基本的数据目录。而iOS系统中,各个应用程序独自维护自己的数据资源和储存文件,在不经用户许可的情况下无法和其他程序交换文件。在iOS中,SandBox的本质是一个文件夹,其路径是固定的,文件夹名字使用UUID(全球唯一标识符)随机分配。如果某个应用程序要求其他应用程序目录下的文件资源时,需要严格的权限检查和用户提醒,合格后才能将该文件以文件副本的方式拷贝到自己的目录下。
			
		\subsection{Android权限控制}
			
			Android是一个权限分隔的操作系统,其中每个应用都有其独特的系统标识(Linux用户ID和组ID)。系统各部分也分隔为不同的标识。Android据此将不同的应用以及应用与系统分隔开来。在默认情况下任何应用都没有权限执行对其他应用、操作系统或用户有不利影响的任何操作。这包括读取或写入用户的私有数据(例如联系人或电子邮件)、读取或写入其他应用程序的文件、执行网络访问、使设备保持唤醒状态等。由于每个Android应用都是在进程沙盒中运行,因此应用必须显式共享资源和数据。它们的方法是声明需要哪些权限来获取基本SandBox未提供的额外功能。应用以静态方式声明它们需要的权限,然后Android系统提示用户同意。
			
			基本Android应用默认情况下不会关联权限,这意味着它无法执行对用户体验或设备上任何数据产生不利影响的任何操作。要利用受保护的设备功能,必须在应用清单中包含一个或多个<uses-permission>标记。在Android操作系统中,系统权限通常以其涉及到的数据性质或使用到的硬件的特性而分为安全或危险的。Android操作系统会判定权限是否涉及到使用者的个人私密信息,或是否存在破坏设备的软硬件环境的可能性。对于操作系统认为安全的权限,一般会直接赋予应用程序而不会触发警告消息;而对于系统认为危险的权限,系统将触发警告或提示信息,明确告知风险和责任,用户同意后方可赋予该应用。
		
		\subsection{iOS权限控制}
			
			iOS的权限管理与Android大致类似。主要的不同点在于,iOS系统中所有软件需要读取的数据都会在读取的瞬间触发系统的强制提示,这个时候用户会看到一个系统提示框询问是否授予软件此项权限。如果用户选否,那么软件将无法获取任何对应的内容。相关授权在进行过第一次询问之后,用户可以在隐私设置中调整软件的对应授权。值得一提的是,越狱之后由于软件可以获取整个机器内置储存的访问权限,从某种程度上来说,软件可以直接读取对应的数据库内容而无需通过API进行访问,这个时候的隐私选项和权限控制也就形同虚设了。与Android系统相比还有些不同在于,iOS系统并不提供对于短信、通话记录等Android系统提供读取的接口,所以有些东西,除非用户越狱,是一定无法被应用读取的。

			
	\section{本章小结}
		本章介绍了现代手机操作系统的权限控制,具体包括基本的安全机制、Android系统和iOS系统的权限管理以及它们对内置传感器的权限控制策略。