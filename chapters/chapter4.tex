\chapter{实验与系统评估}
	
	为了评估AppWrapper工具包的性能,我们尝试了Android商业应用。 实验中使用的应用程序是Google Play韩国市场中33个类别的前三个应用程序,并对表格\ref{tab2}中列出的99个应用程序进行了实验。在实验中,我们使用了三星Galaxy S6和Android 7.0版。 计算机系统使用Windows 7 64位,CPU为Intel i7-4770 3.4 GHz,内存为8 GB。 实验证实,安装并正常执行99 .apk文件的反编译,安全修补和重新打包后创建的.apk文件。 图\ref{fig_5}显示了添加安全功能之前和之后的应用程序启动屏幕。
	
	\begin{table}
		\centering
		\caption{99个样本的分布}
		\label{tab2}
		\begin{tabular}{|c|c|c|}
			\hline
			数量 & 样本大小 & 比例\% \\
			\hline
			36 & < 10 MB & 37\\
			\hline
			19 & 10 – 20 MB & 19\\
			\hline
			16 & 20 – 30 MB & 16\\
			\hline
			11 & 30 – 40 MB & 11\\
			\hline
			8 & 40 – 50 MB & 8\\
			\hline
			9 & > 50 MB & 9\\
			\hline
		\end{tabular}
	\end{table}

	\section{实验结果}
		通过使用AppWrapper工具包为99个应用程序添加安全功能,除了20个安全应用程序和一个Android内置应用程序外,成功率为100%。 对于具有安全功能的应用程序,包含.apk文件以防止修改。 此功能可防止在执行应用程序后在字节码级别进行修改并验证签名密钥。 在前一种情况下,smali文件不会被反编译。 后者在运行应用程序后,应用程序服务器检查签名密钥,生成异常应用程序警告,并关闭应用程序。 Android内置应用程序默认安装在Android上,该应用程序只能安装在Google Play Market上。
		
	\section{处理时间}
	
		\begin{table}
			\centering
			\caption{处理时间}
			\label{tab3}
			\begin{tabular}{|c|c|c|c|c|c|}
				\hline
				App & 反编译 & 安全补丁 & 重编译 & 签名 & 总时间\\
				\hline
				< 10 MB & 7.340 & 1.010 & 18.346 & 2.254 & 28.921\\
				\hline
				10 – 20 MB & 11.994 & 1.516 & 31.351 & 3.951 & 48.946\\
				\hline
				20 – 30 MB & 13.648 & 1.841 & 43.189 & 7.590 & 66.268\\
				\hline
				30 – 40 MB & 17.479 & 2.383 & 55.595 & 10.480 & 85.937\\
				\hline
				40 – 50 MB & 18.068 & 2.662 & 61.248 & 13.886 & 95.864\\
				\hline
				> 50 MB & 16.354 & 1.752 & 52.538 & 15.690 & 86.334\\
				\hline
			\end{tabular}
		\end{table}

		通过将安全功能添加到每个步骤(反编译,安全补丁,重新打包(重新编译,签名))来测量安全功能添加的处理时间。计算与应用程序大小部分对应的每个应用程序的平均处理时间,以确定每步的处理时间。如表\ref{tab3}所示,添加安全可执行代码花费的时间最少。此过程从整个阶段花费的总时间的最大3.49%至少花费2.02%,并且在最少1秒内最多2.6秒内,安全性被添加到AndroidManifest中声明的所有活动方法中。该应用程序的xml文件。

		随着应用程序的大小增加,每个步骤的处理时间通常会增加,并且安全补丁处理时间也会增加。这是因为随着应用程序大小的增加,活动类数量会增加,并且随着更多安全功能的添加,安全补丁时间似乎也会增加。

		在从反编译到重新打包的时间内,应用程序大小为10 MB或更小时所需的最小时间跨度平均为28秒。相反,对于40-50 MB的应用程序大小,最耗时的部分花了大约95秒。这是因为反编译和重新打包所需的时间增加了。安全补丁时间约占总时间的2-3%,并且所提出的框架的安全补丁处理时间并不重要。此外,如果要在以后更改策略,则不需要重新修补安全补丁;因此,重新包装不需要时间。
		
	\section{文件大小}
	
		\begin{table}
			\centering
			\caption{文件大小}
			\label{tab3}
			\begin{tabular}{|c|c|c|c|c|c|}
				\hline
				App & APK大小 & 安全补丁后大小 & 主Class文件大小 & 安全补丁后大小\\
				\hline
				< 10 MB & 6,345KB & 6,538KB & 34KB & 41KB\\
				\hline
				10 – 20 MB & 14,696KB & 14,942KB & 60KB & 70KB\\
				\hline
				20 – 30 MB & 25,737KB & 27,155KB & 25KB & 31KB\\
				\hline
				30 – 40 MB & 34,974KB & 35,553KB & 17KB & 22KB\\
				\hline
				40 – 50 MB & 47,285KB & 47,620KB & 27KB & 35KB\\
				\hline
				> 50 MB & 75,538KB & 73,984KB & 147KB & 165KB\\
				\hline
			\end{tabular}
		\end{table}

		将安全功能添加到AndroidManifest.xml中声明的所有活动中的所有方法后重新打包的.apk文件和主类文件的大小变化并不重要。表4显示了根据.apk文件大小分类的.apk文件大小和主类文件大小的变化。

		首先,小于10 MB的.apk文件大小显示平均文件大小增加3%;但是,主类文件大小增加了15.7%。尽管每个大小的文件大小和主类大小增加率不同,但.apk文件大小增加了0.6%到5.2%,文件的总体平均值显示文件大小增加了2.54%。主类文件的大小增加了10.8%到22.8%。

		尽管应用安全补丁以使各种安全功能能够在活动的所有方法中执行,但与现有的Lee研究相比,文件大小仅增加了2.05%(平均五个办公应用程序,增加0.49%)。很难直接与现有研究的五组进行比较;但是,建议的框架为78个应用程序的所有方法单元增加了安全性,并且文件大小增加最小。这是因为它使用Java反射技术调用安全函数库。通过策略文件控制安全功能也很重要,这样只有最不安全的代码才会添加到不安全应用程序活动中的所有方法中。
